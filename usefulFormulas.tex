%
% Copyright © 2016 Peeter Joot.  All Rights Reserved.
% Licenced as described in the file LICENSE under the root directory of this GIT repository.
%
%\paragraph{Useful identities.}

\index{gradient!radial functions}
%\begin{equation}\label{eqn:emtLecture3:540}
\begin{equation}\label{eqn:usefulFormulas:620}
\spacegrad' \inv{R} = \frac{\rcap}{R^2} = -\spacegrad \inv{R}
\end{equation}
%\begin{equation}\label{eqn:emtLecture3:560}
\begin{equation}\label{eqn:usefulFormulas:640}
\spacegrad R = \rcap = \frac{\BR}{R}
\end{equation}
%\begin{equation}\label{eqn:emtLecture3:580}
\begin{equation}\label{eqn:usefulFormulas:660}
\spacegrad f(R) = \rcap \PD{R}{f}
\end{equation}
%\begin{equation}\label{eqn:emtLecture3:600}
\index{delta function!Laplacian}
\begin{equation}\label{eqn:usefulFormulas:680}
-\spacegrad^2 \inv{R} = 4\pi \delta(\BR)
\end{equation}

\index{curl}
\begin{dmath}\label{eqn:emtLecture1:60}
\spacegrad \cross \Bf
=
\begin{vmatrix}
\Be_1 & \Be_2 & \Be_3 \\
\PDi{x}{} &
\PDi{y}{} &
\PDi{z}{} \\
f_x & f_y & f_z
\end{vmatrix}.
\end{dmath}

\index{Laplacian!decomposition}
\begin{equation}\label{eqn:usefulFormulas:700}
\spacegrad \cross \lr{ \spacegrad \cross \BA } = \spacegrad \lr{ \spacegrad \cdot \BA } - \spacegrad^2 \BA.
\end{equation}

\paragraph{Proofs.}

This result was used in ps1 problem 3,5, and 6.

\index{curl of curl}
\begin{dmath}\label{eqn:emtProblemSet1Appendix:460}
\spacegrad \cross \lr{ \spacegrad \cross \BA }
=
\epsilon_{a b c} \Be_a \partial_b \lr{ \epsilon_{r s t} \Be_r \partial_s A_t }_c
=
\epsilon_{a b c} \Be_a \partial_b \epsilon_{c s t} \partial_s A_t
=
\delta_{ab}^{[st]}
\Be_a \partial_b \partial_s A_t
=
\Be_a \partial_b \lr{ \partial_a A_b - \partial_b A_a }
=
\spacegrad \lr{ \spacegrad \cdot \BA } - \spacegrad^2 \BA.
\end{dmath}

\paragraph{Cylindrical coordinates}

\begin{dmath}\label{eqn:usefulFormulas:720}
\begin{aligned}
\rhocap &= \Be_1 e^{\Be_1 \Be_2 \phi} \\
\phicap &= \Be_2 e^{\Be_1 \Be_2 \phi} \\
\zcap &= \Be_3
\end{aligned}
\end{dmath}

\begin{dmath}\label{eqn:usefulFormulas:740}
\begin{aligned}
\partial_\phi \rhocap &= \phicap \\
\partial_\phi \phicap &= -\rhocap
\end{aligned}
\end{dmath}

\begin{dmath}\label{eqn:usefulFormulas:760}
\spacegrad = \rhocap \partial_\rho + \frac{\phicap}{\rho} \partial_\phi + \zcap \partial_z
\end{dmath}

\begin{dmath}\label{eqn:usefulFormulas:780}
\spacegrad \cdot \BA
=
\inv{\rho} \partial_\rho (\rho A_\rho) + \frac{1}{\rho} \partial_\phi A_\phi + \partial_z A_z
\end{dmath}

\begin{dmath}\label{eqn:usefulFormulas:800}
\spacegrad \cross \BA
=
\rhocap
\lr{
   \inv{\rho} \partial_\phi A_z
   - \partial_z A_\phi
}
+
\phicap
\lr{
   \partial_z A_\rho
   -\partial_\rho A_z
}
+
\inv{\rho} \zcap \lr{
   \partial_\rho ( \rho A_\phi )
   - \partial_\phi A_\rho
}
\end{dmath}

\begin{dmath}\label{eqn:usefulFormulas:820}
\spacegrad^2
=
\inv{\rho} \PD{\rho}{} \lr{ \rho \PD{\rho}{} }
+ \frac{1}{\rho^2} \PDSq{\phi}{}
+ \PDSq{z}{}
\end{dmath}

\paragraph{Spherical coordinates}

\begin{dmath}\label{eqn:usefulFormulas:840}
\begin{aligned}
\rcap &= \Be_1 e^{i\phi} \sin\theta + \Be_3 \cos\theta \\
\thetacap &= \cos\theta \Be_1 e^{i\phi} - \sin\theta \Be_3 \\
\phicap &= \Be_2 e^{i\phi}
\end{aligned}
\end{dmath}

\begin{dmath}\label{eqn:usefulFormulas:860}
\begin{aligned}
\partial_{\theta}{\rcap}      &= \thetacap \\
\partial_{\phi}{\rcap}        &= S_\theta \phicap \\
\partial_{\theta}{\thetacap}  &= -\rcap \\
\partial_{\phi}{\thetacap}    &= C_\theta \phicap \\
\partial_{\theta}{\phicap}    &= 0 \\
\partial_{\phi}{\phicap}      &= -\rcap S_\theta - \thetacap C_\theta.
\end{aligned}
\end{dmath}

\begin{dmath}\label{eqn:usefulFormulas:880}
\spacegrad
=
\rcap \PD{r}{} +
\frac{\thetacap}{r} \PD{\theta}{} +
\frac{\phicap}{r\sin\theta} \PD{\phicap}{}.
\end{dmath}

\begin{dmath}\label{eqn:usefulFormulas:900}
\spacegrad \cdot \BA
=
\inv{r^2} \partial_r (r^2 A_r)
+ \inv{r S_\theta} \partial_\theta (S_\theta A_\theta)
+ \frac{1}{r S_\theta} \partial_\phi A_\phi.
\end{dmath}

\begin{dmath}\label{eqn:usefulFormulas:920}
\spacegrad \cross \BA
=
   \rcap \lr{
       \inv{r S_\theta} \partial_\theta (S_\theta A_\phi)
      -\frac{1}{r S_\theta} \partial_\phi A_\theta
   }
   + \thetacap \lr{
      \frac{1}{r S_\theta} \partial_\phi A_r
      -\inv{r} \partial_r (r A_\phi)
   }
   + \phicap \lr{
        \inv{r} \partial_r (r A_\theta)
      - \inv{r} \partial_\theta A_r
   }
\end{dmath}

\begin{dmath}\label{eqn:usefulFormulas:940}
\spacegrad^2 \psi
=
    \inv{r^2} \PD{r}{} \lr{ r^2 \PD{r}{ \psi} }
   + \frac{1}{r^2 \sin\theta} \PD{\theta}{} \lr{ \sin\theta \PD{\theta}{ \psi } }
   + \frac{1}{r^2 \sin^2\theta} \PDSq{\phi}{ \psi}
\end{dmath}

\paragraph{Vector calculus.}

Enumerate various vc theorems (divergence, curl, the cross product version used in the BC problem, ...)

\paragraph{Normal and tangential decomposition.}

%
% Copyright © 2016 Peeter Joot.  All Rights Reserved.
% Licenced as described in the file LICENSE under the root directory of this GIT repository.
%
%\section{Appendix II.  Problem 3.  Normal and tangential decomposition.}

The decomposition of \cref{eqn:emtProblemSet3Problem3:60} can be derived easily using Geometric Algebra
\index{Geometric Algebra}

\begin{dmath}\label{eqn:emtProblemSet3Problem3:80}
\BA
=
\ncap^2 \BA
=
\ncap (\ncap \cdot \BA)
+\ncap (\ncap \wedge \BA)
%=
%\ncap (\ncap \cdot \BA)
%+
%\ncap \cdot (\ncap \wedge \BA)
\end{dmath}

\index{grade one selection}
The last dot product can be expanded as a grade one (vector) selection

\begin{dmath}\label{eqn:emtProblemSet3Problem3:100}
\ncap (\ncap \wedge \BA)
=
\gpgradeone{
\ncap (\ncap \wedge \BA)
}
=
\gpgradeone{
\ncap I (\ncap \cross \BA)
}
=
I^2 \ncap \cross (\ncap \cross \BA)
=
- \ncap \cross (\ncap \cross \BA),
\end{dmath}

\index{normal component}
\index{tangential component}
so the decomposition of a vector \( \BA \) in terms of its normal and tangential projections is
\begin{dmath}\label{eqn:emtProblemSet3Problem3:120}
\BA
=
\ncap (\ncap \cdot \BA)
-
\ncap \cross (\ncap \cross \BA).
\end{dmath}

I'm not sure how to naturally determine this relationship using traditional vector algebra.  However, it can be verified by expanding the triple cross product in coordinates using tensor contraction formalism

\index{triple cross product}
\begin{dmath}\label{eqn:emtProblemSet3Problem3:140}
-\ncap \cross (\ncap \cross \BA)
=
-\epsilon_{xyz} \Be_x n_y \lr{\ncap \cross \BA}_z
=
-\epsilon_{xyz} \Be_x n_y \epsilon_{zrs} n_r A_s
=
-\delta_{xy}^{[rs]}
\Be_x n_y n_r A_s
=
-\Be_x n_y \lr{ n_x A_y -n_y A_x }
= -\ncap (\ncap \cdot \BA) + (\ncap \cdot \ncap) \BA
= \BA - \ncap (\ncap \cdot \BA).
\end{dmath}

\index{projection}
\index{rejection}
This last statement illustrates the geometry of this decomposition, showing that the tangential projection (or normal rejection) of a vector is really just the vector minus its normal projection.

%This can be rearranged to show that the
%\begin{dmath}\label{eqn:emtProblemSet3Problem3:100}

%The tangential projection, can also be expanded in dot products
%
%\begin{dmath}\label{eqn:emtProblemSet3Problem3:200}
%\ncap (\ncap \wedge \BA)
%=
%\ncap \cdot (\ncap \wedge \BA)
%=
%\BA - \ncap (\ncap \cdot \BA)
%\end{dmath}


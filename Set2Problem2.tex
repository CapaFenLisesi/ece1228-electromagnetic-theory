%
% Copyright � 2016 Peeter Joot.  All Rights Reserved.
% Licenced as described in the file LICENSE under the root directory of this GIT repository.
%
\makeproblem{Infinite line charge.}{emt:problemSet2:2}{
\index{line charge}

An infinitely long straight line charge has a constant charge density \( \rho_l \) [\si{C/m}].

\makesubproblem{}{emt:problemSet2:2a}
Using the integral formulation for \( \BE \)
discussed in the class calculate the electric
field at an arbitrary point \( \BA(\rho, \phi, z) \).
\makesubproblem{}{emt:problemSet2:2b}

Using the Gauss law calculate the same as \partref{emt:problemSet2:2a}.

\makesubproblem{}{emt:problemSet2:2c}

Now suppose that our uniformly charged (\( \rho_l \) constant)
has a finite extension from \( z = a \) to \( z = b \), as sketched in \cref{fig:problemset2:problemset2Fig2}.
\imageFigure{../figures/ece1228-electromagnetic-theory/problemset2Fig2}{Line charge.}{fig:problemset2:problemset2Fig2}{0.3}
Find the electric field at the
arbitrary point \( \BA \).

Note: Express your results in cylindrical coordinate
system.
} % makeproblem

\makeanswer{emt:problemSet2:2}{\withproblemsetsParagraph{
\makeSubAnswer{}{emt:problemSet2:2a}

Since the line charge is of infinite length and rotationally symmetric with respect to \( \phi \), the observation point can be positioned anywhere convenient, such as

\begin{dmath}\label{eqn:emtProblemSet2Problem2:20}
\BA = \rho \rhocap.
\end{dmath}

Let the point on the wire be
\begin{dmath}\label{eqn:emtProblemSet2Problem2:40}
\Br' = z \zcap.
\end{dmath}

The absolute distance between the observation point and the element of charge is
\begin{dmath}\label{eqn:emtProblemSet2Problem2:60}
\Abs{\BA - \Br'} = \sqrt{ (\rho \rhocap)^2 + (z \zcap)^2} = \sqrt{ \rho^2 + z^2 }.
\end{dmath}

The electric field can now be expressed in integral form
\begin{dmath}\label{eqn:emtProblemSet2Problem2:80}
\BE(\BA)
=
\inv{ 4 \pi \epsilon_0 } \int_{-\infty}^\infty dz \rho_l \frac{ \rho \rhocap - z \zcap }{ \lr{\rho^2 + z^2}^{3/2} }
=
\frac{\sigma_l \BA }{ 4 \pi \epsilon_0 } \int_{-\infty}^\infty dz \inv{ \lr{\rho^2 + z^2}^{3/2} }
=
\frac{\sigma_l \BA }{ 4 \pi \epsilon_0 \rho^3 } \int_{-\infty}^\infty dz \inv{ \lr{1 + (z/\rho)^2}^{3/2} }
=
\frac{\sigma_l \BA }{ 4 \pi \epsilon_0 \rho^2 } \int_{-\infty}^\infty du \inv{ \lr{1 + u^2}^{3/2} }.
\end{dmath}

The \( \zcap \) term was killed since the integrand was an odd function in \( z \).  Note that

\begin{dmath}\label{eqn:emtProblemSet2Problem2:100}
   \int \frac{du}{ \lr{1 + u^2}^{3/2} }
= \frac{u}{\sqrt{1 + u^2}},
\end{dmath}

which has a PV limit over \([-\infty,\infty]\) of 2.  This gives

%\begin{dmath}\label{eqn:emtProblemSet2Problem2:120}
\boxedEquation{eqn:emtProblemSet2Problem2:120}{
\BE(\rho)
=
\inv{ 4 \pi \epsilon_0 }
\frac{2 \sigma_l }{\rho} \rhocap.
}
%\end{dmath}

\makeSubAnswer{}{emt:problemSet2:2b}

Using Gauss's law integrating over a cylindrical segment surrounding the wire, we have

\begin{dmath}\label{eqn:emtProblemSet2Problem2:140}
\Delta z \frac{\rho_l}{\epsilon_0}
= \int \spacegrad \cdot \BE dV
= \oint \ncap \cdot \BE dA
= E_\rho(\rho) (2 \pi \rho \Delta z),
\end{dmath}

This assumes the end surfaces at infinity contribute nothing to the surface integral, and gives after rearrangement

\begin{dmath}\label{eqn:emtProblemSet2Problem2:160}
E_\rho(\rho)
= \inv{2 \pi \rho } \frac{\rho_l}{\epsilon_0}.
\end{dmath}

This matches \cref{eqn:emtProblemSet2Problem2:120} as expected.

\makeSubAnswer{}{emt:problemSet2:2c}
\index{cylindrical coordinates}
For the finite problem, there is still rotational symmetry, but the \( \zcap \) term of the field is no longer cancelled out, and we have a \( z \) dependence in the field.

Let
\begin{dmath}\label{eqn:emtProblemSet2Problem2:240}
\begin{aligned}
\BA &= z \zcap + \rho\rhocap \\
\Br' &= z' \zcap,
\end{aligned}
\end{dmath}

for which

\begin{dmath}\label{eqn:emtProblemSet2Problem2:260}
\begin{aligned}
\BA - \Br' &= (z - z') \zcap + \rho \rhocap \\
\Abs{\BA - \Br'}^2 &= (z - z')^2 + \rho^2.
\end{aligned}
\end{dmath}

\begin{dmath}\label{eqn:emtProblemSet2Problem2:180}
\BE(\BA)
=
\frac{\rho_l}{ 4 \pi \epsilon_0 } \int_{a}^b dz \frac{ \rho \rhocap + (z - z') \zcap }{ \lr{\rho^2 + (z-z')^2}^{3/2} }.
\end{dmath}

Let \( z' - z = \rho u \), for
\begin{dmath}\label{eqn:emtProblemSet2Problem2:280}
\BE(\rho, z)
=
\frac{\rho_l}{ 4 \pi \epsilon_0 \rho } \int_{(a-z)/\rho}^{(b-z)/\rho} du \frac{ \rhocap - u \zcap }{ \lr{1 + u^2}^{3/2} }.
\end{dmath}

Note that
\begin{dmath}\label{eqn:emtProblemSet2Problem2:200}
\int du \frac{ u }{ \lr{1 + u^2}^{3/2} } = -\inv{\sqrt{1 + u^2}},
\end{dmath}

so
\begin{dmath}\label{eqn:emtProblemSet2Problem2:220}
\BE(\rho, z) =
\frac{\rho_l }{ 4 \pi \epsilon_0 \rho } \lr{
\evalrange{ \frac{\zcap}{\sqrt{1 + u^2}} }{(a-z)/\rho}{(b-z)/\rho}
+
\evalrange{ \frac{u \rhocap}{\sqrt{1 + u^2}} }{(a-z)/\rho}{(b-z)/\rho}
}.
\end{dmath}

This expands to
%\begin{dmath}\label{eqn:emtProblemSet2Problem2:300}
\boxedEquation{eqn:emtProblemSet2Problem2:320}{
\BE(\rho, z)
=
\frac{\rho_l }{ 4 \pi \epsilon_0 \rho } \lr{
\frac{\rho \zcap + (b-z)\rhocap}{\sqrt{\rho^2 + (b-z)^2}}
-\frac{\rho \zcap + (a-z)\rhocap}{\sqrt{\rho^2 + (a-z)^2}}
}.
}
%\end{dmath}

Observe that we recover \cref{eqn:emtProblemSet2Problem2:120} when \( z = 0, a = -b, b \rightarrow \infty \), as expected.
}}

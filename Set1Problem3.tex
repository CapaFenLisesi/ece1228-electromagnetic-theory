%
% Copyright � 2016 Peeter Joot.  All Rights Reserved.
% Licenced as described in the file LICENSE under the root directory of this GIT repository.
%
\index{solenoidal}
\index{irrotational}
\makeproblem{Solenoidal and irrotational fields.}{emt:problemSet1:3}{
In terms of \(\BE\) or \(\BH\) give an example for each of the following conditions:

\makesubproblem{}{emt:problemSet1:3a}
Field is solenoidal and irrotational.
\makesubproblem{}{emt:problemSet1:3b}
Field is solenoidal and rotational.
\makesubproblem{}{emt:problemSet1:3c}
Field is non-solenoidal and irrotational.
\makesubproblem{}{emt:problemSet1:3d}
Field is non-solenoidal and rotational.
} % makeproblem

\makeanswer{emt:problemSet1:3}{
\makeSubAnswer{}{emt:problemSet1:3a}

We can find a \( \BH \) for which \( \spacegrad \cdot \BH = 0 \) and \( \spacegrad \cross \BH = 0 \).
For simple media where \( \BB = \mu \BH \), let \( \BH = \mu \spacegrad \cross \BA \).  This automatically satisfies \( \spacegrad \cdot \BH = 0 \), and has curl

\begin{dmath}\label{eqn:emtProblemSet1Problem3:20}
\spacegrad \cross \BH
=
\mu \spacegrad \cross ( \spacegrad \cross \BA )
=
\mu \lr{ \spacegrad (\spacegrad \cdot \BA) -\spacegrad^2 \BA} .
\end{dmath}

Seeking a \( \BA \) for which this is zero, a trial function such as \( \BA = \Be_1 \psi \) seems like a reasonable first try, for which we have

\begin{dmath}\label{eqn:emtProblemSet1Problem3:40}
\spacegrad \cross \BH
=
\mu \lr{ \spacegrad (\partial_x \psi) - \Be_1 \spacegrad^2 \psi}.
\end{dmath}

If \( \psi \) is any polynomial in \( y \) or \( z\) that has degrees less than two, this will be zero.  One such function is
\( \BA = A \Be_1 y \), which gives

\begin{dmath}\label{eqn:emtProblemSet1Problem3:60}
\BH =
\mu \begin{vmatrix}
\Be_1 & \Be_2 & \Be_3 \\
\partial_x & \partial_y & \partial_z \\
A f(y) & 0 & 0 \\
\end{vmatrix}
=
-\mu A \Be_3.
\end{dmath}

As there is no positional dependence, this clearly has both \( \spacegrad \cdot \BH = 0 \), and \( \spacegrad \cross \BH = 0 \), the desired respective solenoidal and irrotational properties.

\makeSubAnswer{}{emt:problemSet1:3b}

A field \( \BF \) for which \( \spacegrad \cdot \BF = 0 \) and \( \spacegrad \cross \BF \ne 0 \) is sought.
Any solution to the magnetostatics equations in simple media \( \BB = \mu \BH \) suffices

\begin{dmath}\label{eqn:emtProblemSet1Problem3:80}
\begin{aligned}
   \spacegrad \cross \BH &= \mu \BJ \\
   \spacegrad \cdot \BH &= 0,
\end{aligned}
\end{dmath}

provided \( \BJ \) is non-zero in some region. One such solution is provided by the volume form of the Biot-Savart law

\begin{dmath}\label{eqn:emtProblemSet1Problem3:100}
   \BH = \inv{4\pi} \iiint_V dV \frac{\BJ \cross \rcap'}{\Abs{\Br'}^2}.
\end{dmath}

This has the desired solenoidal
\( \spacegrad \cdot \BH = 0 \)
and rotational
\( \spacegrad \cross \BH \ne 0 \)
properties.

\makeSubAnswer{}{emt:problemSet1:3c}

A field \( \BF \) that is non-solenoidal \( \spacegrad \cdot \BF \ne 0 \) and irrotational \( \spacegrad \cross \BF = 0 \) is sought.
Any solution to the electrostatics equations in simple media \( \BD = \epsilon \BE \) suffices

\begin{dmath}\label{eqn:emtProblemSet1Problem3:120}
\begin{aligned}
   \spacegrad \cdot \BE &= \inv{\epsilon} \rho \\
   \spacegrad \cross \BE &= 0,
\end{aligned}
\end{dmath}

provided \( \rho \) is non-zero in some region.  The standard Coulomb solution provides a field with the desired
non-solenoidal and irrotational properties

\begin{dmath}\label{eqn:emtProblemSet1Problem3:140}
   \BE = \inv{4\pi \epsilon} \iiint_V dV \frac{\rho}{\Abs{\Br}^2}.
\end{dmath}

\makeSubAnswer{}{emt:problemSet1:3d}

A field \( \BF \) that is non-solenoidal \( \spacegrad \cdot \BF \ne 0 \) and rotational \( \spacegrad \cross \BF \ne 0 \) is sought.

Any electric field solution to the general Maxwell's equations for which there is a non-zero charge density, and a magnetic field component has these characteristics.  In simple media where \( \BD = \epsilon \BE \) such a solution will satisfy both Faraday's law and Gauss's law

\begin{dmath}\label{eqn:emtProblemSet1Problem3:160}
\begin{aligned}
   \spacegrad \cross \BE &= -\partial_t \BB \\
   \spacegrad \cdot \BE &= \inv{\epsilon} \rho.
\end{aligned}
\end{dmath}

One such solution is the superposition of an electrostatics solution with any solution describing the propagation of light, for example

\begin{dmath}\label{eqn:emtProblemSet1Problem3:180}
   \BE = \inv{4\pi \epsilon} \iiint_V dV \frac{\rho}{\Abs{\Br}^2} +
\Be_2 E_0 \sin\lr{ \frac{\pi}{a} x } \cos\lr{ \omega t - \beta_\txtz z }
\end{dmath}

This is the superposition of the field from problem 6 with an electrostatics solution.
}

%
% Copyright � 2016 Peeter Joot.  All Rights Reserved.
% Licenced as described in the file LICENSE under the root directory of this GIT repository.
%
\makeproblem{Medium with multiple resonances.}{emt:problemSet5:2}{

Relative permittivity for a medium with multiple resonances is given by:

\begin{equation}\label{eqn:emtProblemSet5Problem2:20}
\epsilon_r = 1 + \chi_e =
1 + \sum_{k=1} \frac{ \omega_{p,k} }{\omega_{0,k}^2 - \omega^2 + j \gamma_k \omega }
\end{equation}

Moreover, the case of an \textit{active medium} (i.e. medium with gain) can be modeled by allowing \( \omega_{p,k} \)
in above to become purely imaginary.
Under these conditions, plot

\begin{dmath}\label{eqn:emtProblemSet5Problem2:40}
\Real\lr{ n(\omega) } -1,
\end{dmath}

and
\begin{dmath}\label{eqn:emtProblemSet5Problem2:60}
\Imag\lr{ n(\omega) }
\end{dmath}
as a function of detuning frequency,

\begin{dmath}\label{eqn:emtProblemSet5Problem2:80}
\nu = \frac{\omega - \omega_c}{2 \pi},
\end{dmath}

for ammonia vapor (an active
medium) where

\begin{dmath}\label{eqn:emtProblemSet5Problem2:100}
\begin{aligned}
\omega_{0,1} &= 2.4165825 \times 10^{15} \si{rad/s} \\
\omega_{0,2} &= 2.4166175 \times 10^{15} \si{rad/s} \\
\omega_{p,k} = \omega_p &= 10^{10} \si{rad/s} \\
\gamma_{k} = \gamma &= 5 \times 10^9 \si{rad/s} \\
\ifrac{(\omega - \omega_c)}{2 \pi} & \in [-7,7] \si{GHz} \\
\omega_c &= 2.4166 \times 10^{15} \si{rad/s}
\end{aligned}
\end{dmath}
} % makeproblem

\makeanswer{emt:problemSet5:2}{

Making the \( \omega_{p,k} \rightarrow j \omega_{p,k} \) transformation toggles the sign of the \(  \omega_{p,k}^2 \) terms in \cref{eqn:emtProblemSet5Problem2:20}, or

\begin{equation}\label{eqn:emtProblemSet5Problem2:41}
\epsilon_r
=
1 - \sum_{k=1} \frac{ \omega_{p,k} }{\omega_{0,k}^2 - \omega^2 + j \gamma_k \omega }.
\end{equation}

The real and imaginary parts of the index of refraction \( n(\omega) = \sqrt{\epsilon_r} \) are plotted as instructed in \cref{fig:p2n:p2nFig3}.

\mathImageFigure{../figures/ece1228-electromagnetic-theory/p2nFig3}{Active medium index of refraction.}{fig:p2n:p2nFig3}{0.4}{ps5:ps5.nb}
}

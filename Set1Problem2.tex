%
% Copyright � 2016 Peeter Joot.  All Rights Reserved.
% Licenced as described in the file LICENSE under the root directory of this GIT repository.
%
\index{field lines!electric}
\makeproblem{Electric field lines.}{emt:problemSet1:2}{
Can either or both of the vector fields shown below represent an electrostatic field \( \BE \). Justify your answer.

\imageTwoFigures
{../figures/ece1228-electromagnetic-theory/emtLect2Fig2}
{../figures/ece1228-electromagnetic-theory/emtLect2Fig3}
{Field lines.}{fig:emtLect2:emtLect2Fig2}{scale=0.3}

} % makeproblem

\makeanswer{emt:problemSet1:2}{\withproblemsetsParagraph{

\begin{enumerate}[(a)]
\item
The first field line configuration looks like it could be the superposition of a set of infinite planes (because of the straight line fields).  Two such planes do not have the degrees of freedom to supply the variability of this field, so let's try three.  Let's assume that we are looking at the configuration sketched in \cref{fig:threePlanes:threePlanesFig3}, where the magnitude (and signs) of the field strengths \( a, b, c \) are not known.

\imageFigure{../figures/ece1228-electromagnetic-theory/threePlanesFig3}{Three infinite planes.}{fig:threePlanes:threePlanesFig3}{0.3}

In regions 1, 2, 3, the field strengths have the following proportionate relationships

\begin{dmath}\label{eqn:emtProblemSet1Problem2:20}
\begin{aligned}
a - b - c &= 1 \\
a + b - c &= 2 \\
a - b - c &= 1
\end{aligned}
\end{dmath}

The solution to these equations is \( (a,b,c) = (1, 1/2, -1/2) \), so this field configuration can be obtained by a sequence of fields with charge densities in this proportion.  The supplied image represents the blue boxed region of the sketch, a dipole configuration with a positive surface charge density twice that of the positive centre surface charge density.

\item
If this second second field configuration is the result of an electrostatic configuration, it is not obvious how.  We can superimpose any number of infinite planes to vary the field strength in regions between the planes, still maintaining the straight line configuration.  However, there's no such superposition that could also vary the field strength side to side from the centre as in the picture.

\end{enumerate}
}}

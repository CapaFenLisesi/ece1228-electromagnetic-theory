%
% Copyright � 2016 Peeter Joot.  All Rights Reserved.
% Licenced as described in the file LICENSE under the root directory of this GIT repository.
%
%\input{../blogpost.tex}
%\renewcommand{\basename}{emt1}
%\renewcommand{\dirname}{notes/ece1228/}
%\newcommand{\keywords}{ECE1228H}
%\input{../peeter_prologue_print2.tex}
%
%%\usepackage{ece1228}
%\usepackage{peeters_braket}
%%\usepackage{peeters_layout_exercise}
%\usepackage{peeters_figures}
%\usepackage{mathtools}
%\usepackage{siunitx}
%
%\beginArtNoToc
%\generatetitle{ECE1228H Electromagnetic Theory.  Lecture 1: Introduction.  Taught by Prof.\ M. Mojahedi}
\chapter{Introduction}
%\label{chap:emt1}
%
%\paragraph{Disclaimer}
%
%Peeter's lecture notes from class.  These may be incoherent and rough.
%
%These are notes for the UofT course ECE1228H, Electromagnetic Theory, taught by Prof. M. Mojahedi, covering \textchapref{{1}} \%citep{balanis1989advanced} content.

\paragraph{Maxwell's equations}
\index{Maxwell's equations!time domain}

\begin{itemize}
\item Faraday's Law
\begin{dmath}\label{eqn:emtLecture1:20}
\spacegrad \cross \BE( \Br, t ) = - \PD{t}{\BB}(\Br, t) - \BM_i
\end{dmath}
\item Ampere-Maxwell equation
\begin{dmath}\label{eqn:emtLecture1:40}
\spacegrad \cross \BH( \Br, t ) = \BJ_\txtc(\Br, t) + \PD{t}{\BD}(\Br, t)
\end{dmath}
\item Gauss's law
\begin{dmath}\label{eqn:emtLecture1:80}
\spacegrad \cdot \BD(\Br, t) = \rho_{\txte\txtv}(\Br, t)
\end{dmath}
\item Gauss's law for magnetism
\begin{dmath}\label{eqn:emtLecture1:100}
\spacegrad \cdot \BB(\Br, t) = \rho_{\txtm\txtv}(\Br, t)
\end{dmath}
\end{itemize}

After unpacking, we have a total of eight equations, with four vectoral field variables, and 8 sources, all interrelated by partial derivatives in space and time coordinates.

It will be left to homework to show that without the displacement current \( \PDi{t}{\BD} \), these equations will not satisfy conservation relations.

The fields are and sources are
\index{units}
\begin{itemize}
\item \( \BE \) Electric field intensity \si{V/m}.
\item \( \BB \) Magnetic flux density \si{V s/m^2} (or Tesla).
\item \( \BH \) Magnetic field intensity \si{A/m}.
\item \( \BD \) Electric flux density \si{C/m^2}.
\item \( \rho_{\txte\txtv} \) Electric charge volume density
\item \( \rho_{\txtm\txtv} \) Magnetic charge volume density
\item \( \BJ_{\txtc} \) Impressed (source) electric current density \si{A/m^2}.  This is the charge passing through a plane in a unit time.  Here \( \txtc \) is for ``conduction''.
\item \( \BM_{\txti} \) Impressed (source) magnetic current density \si{V/m^2}
\end{itemize}

In an undergrad context we'll have seen the electric and magnetic fields in the Lorentz force law

\begin{dmath}\label{eqn:emtLecture1:120}
\BF = q \Bv \cross \BB + q\BE.
\end{dmath}

In SI there are 7 basic units.  These include

\begin{itemize}
\item Length \si{m}.
\item mass \si{kg}.
\item Time \si{s}.
\item Ampere \si{A}.
\index{unit!ampere}
\item Kelvin \si{K} (temperature)
\index{unit!Kelvin}
\item Candela (luminous intensity)
\index{unit!candela}
\item Mole (amount of substance)
\index{unit!mole}
\end{itemize}

\index{unit!Coulomb}
Note that the Coulomb is not a fundamental unit, but the Ampere is.  This is because it is easier to measure.

For homework: show that magnetic field lines must close on themselves when there are no magnetic sources (zero divergence).  This is opposed to electric fields that spread out from the charge.

%\EndNoBibArticle

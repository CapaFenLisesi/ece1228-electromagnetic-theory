%
% Copyright � 2016 Peeter Joot.  All Rights Reserved.
% Licenced as described in the file LICENSE under the root directory of this GIT repository.
%
%{
\makeproblem{Dipole moment density for disk.}{emt:problemSet2:4}{
\index{dipole moment density}
A dielectric circular disk of radius \( a \) and thickness \( d \) is permanently polarized with a
dipole moment per unit volume \( \BP \) [\si{C/m^2}], where \( \Abs{\BP} \)
is
constant and parallel to the disk axis (z-axis here) as
shown in \cref{fig:problemset2:problemset2Fig3}.

\imageFigure{../../figures/ece1228-emt/problemset2Fig3}{Circular disk geometry.}{fig:problemset2:problemset2Fig3}{0.3}

\makesubproblem{}{emt:problemSet2:4a}
Calculate the potential along the disk axis for \( z > 0 \).
\makesubproblem{}{emt:problemSet2:4b}
Approximate the result obtained in \partref{emt:problemSet2:4a} for the case
of \( Z \gg d \).
} % makeproblem

\makeanswer{emt:problemSet2:4}{
\makeSubAnswer{}{emt:problemSet2:4a}

In class the potential for a discrete dipole pair was found to be

\begin{dmath}\label{eqn:emtProblemSet2Problem4:20}
V(\Br) = \inv{4 \pi \epsilon_0} \frac{\rcap \cdot \Bp}{r^2},
\end{dmath}

so in the continuum, the element of dipole moment is \( d\Bp = \BP(\Br') dV' \), for a total potential of

\begin{dmath}\label{eqn:emtProblemSet2Problem4:40}
V(\Br)
= \inv{4 \pi \epsilon_0} \int dV' \frac{\lr{ \Br - \Br'} \cdot \BP(\Br')}{\lr{\Br - \Br'}^3}.
\end{dmath}

Restricting \( \Br \) to the z-axis, with \( \Br(z) = z \zcap \), and using cylindrical coordinates for the disk

\begin{dmath}\label{eqn:emtProblemSet2Problem4:60}
\Br' = z'\zcap + \rho'\rhocap,
\end{dmath}

where \( z' \in [-d,0] \) and \( \rho' \in [0, a] \).  The difference vector between the charge and the observation point is

\begin{dmath}\label{eqn:emtProblemSet2Problem4:80}
\Br - \Br' = (z - z')\zcap - \rho' \rhocap,
\end{dmath}

with magnitude
\begin{dmath}\label{eqn:emtProblemSet2Problem4:100}
\Abs{\Br - \Br'}^2 = (z - z')^2 + (\rho')^2.
\end{dmath}

With \( \BP = \Abs{\BP} \zcap \), the potential along the axis is

\begin{dmath}\label{eqn:emtProblemSet2Problem4:120}
V(z)
=
\frac{2 \pi \Abs{\BP} }{4 \pi \epsilon_0}
\int_0^a \rho' d\rho \int_{-d}^0 dz' \frac{\lr{ (z - z')\zcap - \rho' \rhocap} \cdot \zcap}{\lr{(z - z')^2 + (\rho')^2}^{3/2}}
=
\frac{\Abs{\BP} }{2 \epsilon_0}
\int_0^a \rho' d\rho \int_{-d}^0 dz' \frac{ (z - z') }{\lr{(z - z')^2 + (\rho')^2}^{3/2}}
=
% u = z' - z
% v = \rho'
-\frac{\Abs{\BP} }{2 \epsilon_0}
\int_0^a dv \int_{-d-z}^{-z} du \frac{ u v }{\lr{u^2 + v^2}^{3/2}}
=
\frac{\Abs{\BP} }{2 \epsilon_0}
\int_{-d-z}^{-z} du \evalrange{\lr{\frac{ u }{\lr{u^2 + v^2}^{1/2}}}}{v =0}{a}
=
\frac{\Abs{\BP} }{2 \epsilon_0}
\int_{-d-z}^{-z} du \lr{ \frac{ u }{\sqrt{u^2 + a^2}} - \frac{u}{\Abs{u}} }
=
\frac{\Abs{\BP} }{2 \epsilon_0}
\int_{-d-z}^{-z} du \lr{ \frac{ u }{\sqrt{u^2 + a^2}} - \sgn{u}}.
\end{dmath}

Looking at the values of the integration variable \( u \), note that
for \( z > 0 \), \( u < 0 \) and for \( z < -d \), \( u > 0 \), so the potential outside of the disk is

\begin{dmath}\label{eqn:emtProblemSet2Problem4:140}
V(z)
=
\frac{\Abs{\BP} }{2 \epsilon_0} \lr{ d \sgn(z) + \int_{-d-z}^{-z} du \frac{ u }{\sqrt{u^2 + a^2}} }
=
\frac{\Abs{\BP} }{2 \epsilon_0} \lr{ d \sgn(z) +
\evalrange{\lr{\sqrt{u^2 + a^2}}}{-d-z}{-z}
},
\end{dmath}

or

%\begin{dmath}\label{eqn:emtProblemSet2Problem4:160}
\boxedEquation{eqn:emtProblemSet2Problem4:180}{
V(z)
=
\frac{\Abs{\BP} }{2 \epsilon_0} \lr{ d \sgn(z) +
\sqrt{ z^2 + a^2 } - \sqrt{ (d+z)^2 + a^2 }
}.
}
%\end{dmath}

\makeSubAnswer{}{emt:problemSet2:4b}

Let

\begin{dmath}\label{eqn:emtProblemSet2Problem4:200}
f(z) = \sqrt{ z^2 + a^2 },
\end{dmath}

which has a derivative of
%This has a derivative that is zero at the origin

\begin{dmath}\label{eqn:emtProblemSet2Problem4:220}
f'(z) = \frac{z}{\sqrt{z^2 + a^2}}.
\end{dmath}

%The second derivative is
%\begin{dmath}\label{eqn:emtProblemSet2Problem4:240}
%f''(z)
%= \frac{1}{\sqrt{z^2 + a^2}} + (-1/2) \frac{ z(2z) }{\lr{z^2 + a^2}^{3/2}}
%= \frac{ a^2 }{\lr{z^2 + a^2}^{3/2}}.
%\end{dmath}
%
The first order Taylor approximation is

\begin{dmath}\label{eqn:emtProblemSet2Problem4:260}
f(z + d) - f(z)
\approx f'(z) d
=
\frac{z d}{\sqrt{z^2 + a^2}},
\end{dmath}

so for \( \Abs{z} \gg d \) the potential is approximated by

\begin{dmath}\label{eqn:emtProblemSet2Problem4:280}
%\boxedEquation{eqn:emtProblemSet2Problem4:300}{
V(z)
\approx
\frac{\Abs{\BP} }{2 \epsilon_0} \lr{ d \sgn(z) - \frac{z d}{\sqrt{z^2 + a^2}} },
%}
\end{dmath}

or
%\begin{dmath}\label{eqn:emtProblemSet2Problem4:360}
\boxedEquation{eqn:emtProblemSet2Problem4:380}{
V(z)
=
\frac{\Abs{\BP} d \sgn(z)}{2 \epsilon_0} \lr{ 1 -
\frac{1}{\sqrt{1+ (a/z)^2}}
}.
}
%\end{dmath}

Note that if \( \Abs{z} \gg a \) too, this can be further approximated as

%\boxedEquation{eqn:emtProblemSet2Problem4:340}{
\begin{dmath}\label{eqn:emtProblemSet2Problem4:320}
V(z)
\approx
\frac{\Abs{\BP} d \sgn(z)}{2 \epsilon_0} \lr{ 1 - (1 + (-1/2) (a/z)^2) }
=
\inv{4 \pi \epsilon_0} \frac{\Abs{\BP} \sgn(z) (d \pi a^2)}{z^2}.
\end{dmath}
%}

This is a slightly tidier result that shows the asymptotic inverse-square \( z \) dependence of the potential more clearly than \cref{eqn:emtProblemSet2Problem4:380}, a result that assumed \( z \gg d \), but did not assume \( z \gg a \).  We also have the volume of the disk as an explicit factor in this approximation.
%, but requires \( z \\gg a,d \), a more strict limiting value than the \( z \gg d \) requested in the problem.
}
%}
